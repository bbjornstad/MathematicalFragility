As of September 14 and termination of the Northwestern Summer Undergraduate Research Program, I have
completed some investigation into the return times and hitting times from initial position $n$ of the
power-distributed birth and death chain. This leaves open the next steps of this research. A number of
questions remain unanswered. Do monotonically increasing transition probability functions $p(x)$ result
in hitting time functions that resemble those generated by the power-distributed chain? How do we
compare chains in which the mean hitting time is not defined (in other words, how do we compare chains
that seem to ``get lost'')?

Perhaps the most important next step of research will involve generalizing some of the results gathered
here. The power-distributed chain is but one chain out of an infinite number of possibilities, and
finding the best way to compare arbitrary chains is critical to developing a complete theory of
mathematical fragility. It may make the most sense to investigate the probability that a birth and death
chain with initial position $n$ and step size $s$ will reach the origin, as this quantity should be
defined for all chains. This will allow comparison of chains where the mean hitting time diverges, like
in the case of the power-distributed chain when $\beta$ is larger than $1$.

To this end, continuing to update simulations to test new parameters and transition probabilities will
help guide intuitions. However, the desire to move into cases of more generality will need more analytic
methods. Hopefully I can continue to develop my investigation into more complex forms of analysis,
perhaps making heavy use of sequences and limiting behavior. While the work presented here is just the
beginning of development of mathematical fragility, I hope to broaden my investigation to generate more
interesting results. Further, I hope that this project will be able to follow my mathematical education.
As I continue to learn techniques from various different fields, I would like to apply these to this
project.
