\documentclass[12pt,letterpaper]{article}
\usepackage[margin=1in]{geometry}
\usepackage[style=numeric,citestyle=numeric-comp,backend=bibtex]{biblatex}
\usepackage{sectsty}
\usepackage{titlesec}

\sectionfont{\large}

\titlespacing\section{0pt}{12pt plus 4pt minus 2pt}{2pt plus 2pt minus 2pt}

\addbibresource{references.bib}

% \bibliography{references}

\begin{document}
    \section*{Introduction}
    Intuitively, the fragility of a system gives an indication of how that system is likely to respond
    in unstable situations. In economics the concept of fragility as it relates to markets is well
    studied \cite{anand}. Fragility in general has been developed by Nassim Taleb, who defines it as
    ``accelerating sensitivity to a harmful stressor'' \cite{taleb2}. Taleb uses this definition to
    explore his concept of ``antifragility'' \cite{taleb1}. However, a survey of the literature reveals
    few works apply a general definition of fragility to specific mathematical models. This is where I
    wish to focus my attention. I intend to formalize a mathematical conception of a system's
    sensitivity to stressors (error) and determine a mathematically precise definition of fragility by
    exploring systems of random walks. I conjecture that fragility is a relation between efficiency and
    error, and in random walks step size represents fragility while the density probability function
    represents error.
    
    \section*{Background}
    The paradigmatic random walk is the \emph{drunkard's walk}. Imagine that an individual leaves a bar
    without knowing where he wishes to go, and so instead decides to flip a coin at each intersection.
    Say the walker begins at 42nd Street. If he flips heads, he will walk to 43rd Street, and if he
    flips tails, he will instead walk to 41st Street.  At each successive intersection, he will perform
    the same routine. His path through the city is determined by a sequence of random coin flips
    \cite{weiss}. This scenario is modeled by a \emph{random walk}. A random walk is at its simplest a
    sum of a number of random variables \cite{weiss}. Random walks are important objects of study in
    mathematics due to their ability to model a large class of non-deterministic systems, with
    applications ranging from physics to the physiology of plants, and are used in economics to model
    market fluctations \cite{stochastic, ibe}. While the framework of a random walk is relatively
    simple, the ability to model a diverse range of systems demonstrates their importance. Developing a
    mathematically precise model of fragility built on systems of random walks is thus developing a
    precise model of fragility that can be applied to other systems as well.
    
    \textbf{Proof vs. Evidence:} In mathematics, a distinction is drawn between evidence and proof.
    Evidence develops intuitions by providing a sense of what is likely to be true. For example, the
    data and patterns discovered through some mathematical methods, such as numerical simulation---a
    tool I will utilize---give an intuition about how a model responds under certain constraints. In
    order to fully develop a mathematical model, properties must be proved---derived logically from a
    set of true assumptions. This contrasts other fields in which evidence does not only provide a sense
    of what is likely to be true, but also the foundation of what is taken to be true. Thus, my
    mathematical model will be developed in two distinct parts: a phase devoted to developing intuitions
    about the system through less rigorous processes, followed with a phase devoted to proving intuitive
    properties of the model in rigorous settings. Numerical simulations will give intuitions, and proofs
    will give proof.

    \section*{Methodology}
    I will start by considering a fixed scenario of a one-dimensional random walk with two parameters: a
    walker steps along a straight line forwards or backwards, hoping to reach the origin. However, the
    walker's vision is obscured by a ``fog'' of varying density. If the density of the fog is higher,
    then the walker will be more likely to incorrectly step away from his destination. This model has
    two parameters that can be varied: the density of the fog, and the size of the walker's steps.

    There are two major questions that I would like to answer through my work. How does the average
    amount of time required for the walker to reach his destination depend on the function governing the
    density of the fog? And how does the walker's success in reaching the destination vary with respect
    to the size of his step? I conjecture that larger step sizes will result in a more ``fragile''
    system, and that probability density represents the error in the random walk. Thus, if I create a
    two parameter random walk, where one parameter affects the error rate, and the other affects the
    fragility of the system, I can investigate how changes in error will affect the system's fragility.

    \textbf{The first four weeks:} I will develop software to numerically estimate the walker's response
    to changes in these two parameters. I will use the MatLab and Python (equipped with the NumPy
    library) environments to iteratively simulate the walker under a variety of functions governing
    probability density. I will explore fogs governed by power decays, exponential decays, logarithmic
    decays, and other potential decay types. The goal is to find families of probability density
    functions that provoke interesting responses from the walker, further developing my intuitions about
    potential properties of this new model of fragility. To do so, I will use tools available in MatLab
    and the NumPy library to quickly analyze and visualize probability functions and the walker's
    responses.
    
    \textbf{The remaining four weeks:} I will attempt to mathematically prove the patterns that appeared
    in my numerical experiments. Due to the distinction drawn between intuitions and proofs, I
    need a mathematically rigorous framework to generate proofs. I will use \emph{analytical}
    methods---those used in the field of analysis---to prove these intuitive results. A variety of tools
    may be useful throughout this phase of work. These methods will include established techniques from
    \emph{Stochastic Analysis}, with emphasis on \emph{Markov Chains} \cite{ibe}. In this phase, I will
    familiarze myself with Markov Chains through literary review. Then, I will use existing research on
    \emph{birth and death processes}---which describe ``phenomena where the variation of a random
    characteristic occurs by discrete jumps''---to frame the random walk as a Markov Chain
    \cite{stochastic}. I will further leverage existing research on these Markov Chains to estimate
    \emph{hitting times} for this random walk \cite{stochastic}.

    \textbf{If time permits:} I expect that these two phases will span a total of eight weeks. However,
    if possible, I will try to generalize my one-dimensional model to higher dimensional spaces.
    Typically generalizations to higher dimensions prove difficult but also enlighten nuances not
    encountered in lower dimensions.

    \section*{Qualifications}
    I am a Mathematics and Computer Science double major as well as an ISP student, and on conclusion of
    the current academic year, I will have completed the MENU undergraduate sequence in Analysis. This
    provides my foundation for proof based research, and my experience with computer science will
    provide me the necessary knowledge to perform numerical simulations. I spent the Summer of 2016
    developing an analysis pipeline for use with large sets of genetic data in the R programming
    environment. Since I intend to purse a career in academia, this project will be a valuable learning
    experience for both my mathematical knowledge and my future academic goals.

    \printbibliography
\end{document}
