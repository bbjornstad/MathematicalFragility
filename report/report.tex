\documentclass[12pt,letterpaper]{article}
\usepackage[margin=1in]{geometry}
\usepackage[style=numeric,citestyle=numeric-comp,backend=bibtex]{biblatex}
\usepackage{sectsty}
\usepackage{titlesec}
\usepackage{amsmath}

\sectionfont{\large}

\titlespacing\section{0pt}{12pt plus 4pt minus 2pt}{2pt plus 2pt minus 2pt}

% \bibliography{references}

\begin{document}
\section*{Introduction}
Fragility describes how a system is likely to respond to small perturbations to its parameters.
I investigated fragility in biased random walks. I considered stochastic models such as birth
and death chains that have the potential to describe a variety of scenarios. I made heavy use of
computer simulation to guide my intuitions, and applied analytical methods. In this writeup, I reflect
on my findings and the process of math research.


\section*{Project Phases}
\noindent
\textbf{Mathematical Context:} At first, my conception of mathematical fragility was vague. This was due
to my inexperience with probability theory. I needed a better understanding of the context, especially
stochastic models. I worked through exercises and referred to a number of textbooks, introducing myself
to the key concepts of recurrence and hitting-time. This took longer than originally expected, yet
proved critically important as I would have been left without appropriate language or definitions
necessary to consider birth and death chains.

\noindent
\textbf{Initial Simulations:} I next turned my attention towards some specific probability
distributions. In particular, I needed to develop my intuitions about birth and death chains, and the
property of recurrence---which describes the chain's potential to return to a certain point. I used
Python with the \texttt{NumPy} and \texttt{Matplotlib} libraries to simulate birth and death chains
defined by various probability distributions, and analyze and plot return times to $0$. I had previous
experience with similar forms of data analysis, and found that this came naturally to me.  However, I
found it challenging to set appropriate parameters for each simulation---too many data points caused
simulations to run unnecessarily long, but too few (or poor sampling technique) produced inconclusive
results. I learned techniques that can be used to investigate hypotheses about return times.

In particular, my first simulations led me to define the \emph{power-distributed chain with parameter
$\beta$}, whose behavior is governed by the probability functions $p(n)$ and $q(n)$. When walking on the
natural numbers, $p(n)$ represents the probability of the chain stepping in the positive direction, and
$q(n)$ the negative. These functions are defined by
\[
    p(n) = \begin{cases}
        1 & \text{if } n = 0 \\
        \frac{1}{2+n^{-\beta}} & \text{if } n > 0
    \end{cases}, \quad \quad
    q(n) = 1-p(n).
\]
I hypothesized that the power-distributed chain is \emph{positive recurrent} (i.e. returns to $0$
infinitely often and in finite expected time) if $\beta < 1$, and is \emph{null recurrent} (i.e.
returns to $0$ infinitely often but with infinite expected return time) if $\beta \geq 1$. Simulations
allowed me to compare return times values of $\beta$, and the results supported my hypothesis.

\noindent
\textbf{Analytic Proof:} In mathematics, however, numerical results from simulation do not constitute
sufficient evidence. As such, I turned to analytic methods to prove my conjecture about the
power-distributed chain. While I drew upon established results to prove that some infinite sums
converge, I needed to be creative with how I applied approximations. I often became frustrated feeling
like I was working in circles. Completing this phase required my persistence as a mathematician.
Eventually, I was able to make some key observations that allowed me to succeed in proving my
conjecture.  Overall, the time spent here felt too long for the results established, but it challenged
both my mathematical ability and my perserverance.

\noindent \textbf{The Right Question:} Through my analytic investigation of the power-distributed birth
and death chain, I had to consider if recurrence properties offered the best information, or if other
properties about birth and death chains would be more enlightening. This led me to consider the more
general \emph{mean hitting time}---the expected number of steps for a chain in a particular state to
reach a different state. Mean hitting times are more general than recurrence times and better represent
a system's ``efficiency'' in reaching a particular state, while recurrence times are a more direct
representation of whether a system frequently ``gets lost.'' I am still unsure that mean hitting times
should be the object of study.

\noindent
\textbf{Exploration of Hitting Times:} Let $h_s(n)$ be the expected hitting time to $0$ from position
$n$ with step size $s$. I was advised that calculating hitting times was difficult, and so investigated
using alternative methods. I returned to simulations, updating the necessary parameters to instead vary
the step-size of the power-distrbuted chain. This helped enlighten quantities of particular interest,
such as $\partial\, h/ \partial\, s$, the change in hitting time with respect to step-size. I believe
that these quantities hold important information about the fragility of a system.


\section*{Further Research}
The work completed thus far has led me to investigate hitting times. I think that hitting times are
still valuable in uncovering more about mathematical fragility, especially as it relates to step-size.
A number of quantities---such as specific limits or partial derivatives---may illuminate facts about how
a birth and death chain responds to perturbations in its parameters. Further work to determine the
specific shapes and classes of hitting time functions $h_s(n)$ for various probability distributions
will help understanding the restrictions of these hitting time functions. This could, eventually, allow
for complicated analytical methods to deduce properties of these functions.

However, it could be that a different quantity offers more suitable information. For example, it might
be more enlightening to consider the probability that a birth and death chain reaches a particular
point. This may allow a broader class of probability distributions to be considered and as such could
direct further research. I expect direct computation of these probabilities to be difficult, and so
further simulation may be the best way to uncover preliminary information about these quantities.


\section*{Final Thoughts}
Above all, this project helped me to better understand my strengths and limits as a mathematician. It
gave me the invaluable opportunity to experience research early in my career. I found it difficult to
predict how long answering a mathematical research question would take, and I had to persist even when I
felt behind schedule or lacking in ideas. I found research very different from my academic work, as I
had to creatively look for new directions. While the answer may not be known when working on problem
sets, I usually find I know the angle to best tackle the problem. In contrast, this research was
challenging in that I both needed to ask the right questions to best guide my ideas, as well as deduce
the best techniques and approaches to use when answering those questions. This is not easy, and gaining
this experience so early in my career will surely aid me in the future. I would like to thank
Northwestern and the Office of Undergraduate Research for giving me this opportunity, as well as
Dr.~Jason Siefken for his incredible mentorship and support throughout this project, even when separated
by great distances.
\end{document}
