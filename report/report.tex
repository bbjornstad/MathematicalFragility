\documentclass[12pt,letterpaper]{article}
\usepackage[margin=1in]{geometry}
\usepackage[style=numeric,citestyle=numeric-comp,backend=bibtex]{biblatex}
\usepackage{sectsty}
\usepackage{titlesec}
\usepackage{amsmath}

\sectionfont{\large}

\titlespacing\section{0pt}{12pt plus 4pt minus 2pt}{2pt plus 2pt minus 2pt}

% \bibliography{references}

\begin{document}
\section*{What I Did}
At first, the conception of mathematical fragility seemed quite vague. This was due to relative
inexperience with probability theory, the underlying mathematical framework that supports the models
considered here. As a result, it was necessary for me to familiarize myself with the mathematical
context in which this project was to take place. In particular, I needed a better understanding of
stochastic models, especially the random walk and Markov chains. To do so, I drew upon Kobayashi, Mark,
and Turin's \emph{Probability, Random Processes, and Stastical Analysis}. Working through exercises in
this textbook helped me become more comfortable working with Markov chains, and introduced me to the key
concepts of recurrence and hitting-times. Further, I applied methods learned in my Analysis classes to
consider such models in abstract probability spaces.

After I felt sufficiently familiarized with critical concepts, I turned my attention towards
investigation of some specific probability distributions. To do so, I utilized a subset of Markov chains
known as Birth and Death chains, where the probability distribution is allowed to vary based on the
current position. This was a reasonable choice due to the fact that I wanted to model the system of a
person walking amidst fog concentrated about a specific point, and such fog should have a dependence on
position. For initial simplicity, I focused on a birth and death chain in which the possible positions
could be the natural numbers. In the scenario focused on the walker, I also made the assumption that the
probability should tend towards one-half as he steps further away from the point of interest (in this
case, the origin). Using these assumptions, I deduced a specific chain of interest, which I denote the
\emph{power-distributed birth and death chain}, defined by the following probability distribution:
\[
    p(n) = \begin{cases}
        1 & \text{if } n = 0 \\
        \frac{1}{2+n^{-\beta}} & \text{if } n > 0
    \end{cases},
\]
and $q(n) = 1-p(n)$. Here, $p(n)$ represents the probability of the chain stepping in the positive
direction, and $q(n)$ is correspondingly the negative. In particular, $\beta$ is a parameter of this
distribution, and can take positive real values.

In order to investigate interesting behavior of the power-distributed chain, I used Python, equipped with
the \texttt{NumPy} and \texttt{Matplotlib} libraries to run simulations, collect data, and visualize
recurrence properties of the power-distributed chain. This helped to further guide my intuitions about
the behavior of this chain, and thus aided me in analyzing the chain using rigorous mathematical
methods.

From here, I sought to better conceptualize where might \emph{fragility} falls in relation to the model
of a birth and death chain on the natural numbers. To do so required investigation of the \emph{hitting
time} from an initial position---which represents the expected number of steps required to reach zero
from a given initial point. In some sense, this represents the ``efficiency'' of the system, as it
quantifies how quickly the probability distribution guides the system to our point of interest. As such,
I turned my attention again to Python to perform simulations in which a new parameter was introduced to
the power-distributed chain---step-size. These simulations generated data under a variety of step-sizes,
initial positions, and values of $\beta$.

However, mathematics requires more than the study of one specific model, and so I began to branch out my
analytical investigation of hitting times. Using an important known relationship between consecutive
hitting-times and the probability distribution of an arbitrary birth and death chain, I was able to
study cases in which the hitting times have special properties. For example, it was natural to question
the existence of a chain in which the hitting time is simply the initial position.

\section*{What I Learned}
Throughout my initial investigation of the power-distributed birth and death chain, simulations led me
to hypothesize that this chain is \emph{positive-recurrent} if $\beta < 1$ and \emph{null-recurrent} if
$\beta > 1$. This is to say that the power-distributed chain can be expected to return to the origin
both infinitely often as well as with finite time of return whenever $\beta < 1$. Similarly, the
power-distributed chain can be expected to return to the origin infinitely often but with infinite time
of return whenever $\beta > 1$. Drawing upon results from \emph{Probability, Random Processes, and
Statistical Analysis}, as well as some from Zhong Li, I managed to prove this hypothesis analytically.
This both solidified my understanding of the behavior of this particular chain, as well as gave me more
opportunity to use analytic methods in the context of probability-theory.

Similarly, simulations guided my intuition about how the hitting times of power-distributed chain
respond to perturbations in the step-size. This allowed me to investigate the existence of chains that
produce interesting hitting-times. Let $h_s(n)$ represent the expected hitting time to zero from initial
position $n$ under step-size $s$. The following relationship proved critical:
\[
    h_s(n) = p(n)(1+h_s(n+s)) + q(n)(1+h_s(n-s)).
\]
In particular, I was interested in questions such as: is there a chain with probability distribution
defined by $p(n)$ so that its associated hitting time is $h_s(n) = n$? What about a similar example
where $h_s(n) = n/s$?

However, perhaps the most useful aspect of what I have learned throughout this project is a better
understanding of the processes, habits, and techniques to use when undertaking math research.
Mathematics research is somewhat special in that it does not require a laboratory, and can be done in
most circumstances. On the one hand, this provides welcome flexibility and a freedom not afforded by
other fields, but on the other this necessitates more discipline and persistence. Predicting a timeline
is especially difficult for mathematics research due to the emphasis on individual work and the high
level of abstraction. As such, my work on this project has led me to better understand my limits as a
mathematician---and thus understand where I can improve---as well as developed persistence to keep
thinking, even if few results are being produced.

\section*{Further Research}
Section forthcoming.
\end{document}
