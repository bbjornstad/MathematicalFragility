\documentclass[12pt,letterpaper]{article}
\usepackage[margin=1in]{geometry}
\usepackage[style=numeric,citestyle=numeric-comp,backend=bibtex]{biblatex}
\usepackage{sectsty}
\usepackage{titlesec}
\usepackage{amsmath}

\sectionfont{\large}

\titlespacing\section{0pt}{12pt plus 4pt minus 2pt}{2pt plus 2pt minus 2pt}

% \bibliography{references}

\begin{document}
\section*{Introduction}
Fragility describes how a system is likely to respond to small perturbations to its parameters.
I investigated fragility in biased random walks. I considered stochastic models such as birth
and death chains that have the potential to describe a variety of scenarios. I made heavy use of
computer simulation to guide my intuitions and applied analytic methods. In this writeup, I reflect
on my findings and the process of math research.


\section*{Project Phases}
\noindent
\textbf{Mathematical Context:} At first, my conception of mathematical fragility was vague. This was in
part due to my inexperience with probability theory. I needed a better understanding of the context,
especially stochastic models. I worked through exercises and referred to a number of textbooks,
introducing myself to the key concepts of \emph{recurrence} and \emph{hitting time}. This took longer
than originally expected, yet laid the mathematical foundation of my project.

\noindent
\textbf{Initial Simulations:} I next turned my attention towards some specific probability
distributions. In particular, I needed to develop my intuitions about birth and death chains, and the
property of recurrence---which describes the chain's potential to return to a certain point. I used the
Python programming language with the \texttt{NumPy} library to simulate birth and death chains defined
by varying transition probabilities. Later, I used the \texttt{Matplotlib} library to plot return times
to zero. I had previous experience with similar forms of data analysis and found that this came
naturally to me.  However, I found it challenging to set appropriate parameters for each
simulation---too many data points caused simulations to run unnecessarily long, but too few (or poor
sampling techniques) produced inconclusive results.

In particular, my first simulations led me to define the \emph{power-distributed chain with parameter
$\beta$}, whose behavior is governed by the probability functions $p(n)$ and $q(n)$. When walking on the
natural numbers, $p(n)$ represents the probability of the chain stepping in the positive direction, and
$q(n)$ the negative. These functions are defined by
\[
    p(n) = \begin{cases}
        1 & \text{if } n = 0 \\
        \frac{1}{2+n^{-\beta}} & \text{if } n > 0
    \end{cases}, \quad \quad
    q(n) = 1-p(n).
\]
I hypothesized that the power-distributed chain is \emph{positive recurrent} (i.e. returns to zero
infinitely often and with finite expected time) if $\beta < 1$, and is \emph{null recurrent} (i.e.
returns to $0$ infinitely often but with infinite expected return time) if $\beta \geq 1$. Simulations
allowed me to compare return times and values of $\beta$. The simulation supported my hypothesis.

\noindent
\textbf{Analytic Proof:} In mathematics, however, numerical results from simulation do not constitute
sufficient evidence. As such, I turned to analytic methods to prove my conjecture about the
power-distributed chain. While I drew upon established results to prove that some infinite sums
converge, I needed to be creative with how I applied approximations. I often became frustrated feeling
like I was working in circles. Completing this phase required my persistence as a mathematician.
Eventually, I was able to make some key observations that allowed me to succeed in proving my
conjecture.  Overall, the time spent here felt too long for the results established, but it challenged
both my mathematical ability and my perserverance.

\noindent \textbf{The Right Question:} Through my analytic investigation of the power-distributed birth
and death chain, I had to consider if recurrence properties offered the best information, or if other
properties about birth and death chains would be more enlightening. This led me to consider the more
general \emph{mean hitting time}: the expected number of steps for a chain in a particular state to
reach a different state. Mean hitting times are more general than return times and better represent
a system's ``efficiency'' in reaching a particular state, while recurrence times are a more direct
representation of whether a system frequently ``gets lost.'' I am still unsure that mean hitting times
are the best object of study.

\noindent
\textbf{Exploration of Hitting Times:} Let $h_s(n)$ be the expected hitting time to $0$ from position
$n$ with step size $s$. I was advised that calculating hitting times was difficult and so investigated
using alternative methods. I returned to simulations, updating the necessary parameters to instead vary
the step size of the power-distrbuted chain. This showed me how quantities of interest, such as
$\partial\, h/ \partial\, s$, vary with respect to step size. I believe that these quantities can be
used to classify the fragility of a random walk.


\section*{Further Research}
More investigation of hitting times of the power-distributed chain remains to be done. The results I
have gathered so far show that the hitting time of the power-distributed chain decreases with changes in
step size. Further, simulations also reveal sequences of step sizes where the mean hitting time
instantaneously drops. I suspect that these sequences are closely related to the division algorithm, and
analysis of these sequences may give information about the shape of hitting time functions.

However, many choices of transition probabilities lead to markov chains with infinite mean hitting
times. Thus, mean hitting times cannot be used to compare these chains. Hitting times give good
information in the case of the power-distributed chain but this will not generalize to chains with
different transition probabilities. Instead, I would like to investigate another quantity: the
probability that the chain reaches a particular point. This would allow comparison of chains with
infinite mean hitting times, as this probability is well-defined even for chains with infinite mean
hitting times. This generalization is critical to developing a complete conception of mathematical
fragility.


\section*{Final Thoughts}
Above all, this project helped me to better understand my strengths and limits as a mathematician. It
gave me the invaluable opportunity to experience research early in my career. I found it difficult to
predict how long answering a mathematical research question would take, and I had to persist even when I
felt behind schedule or was lacking in ideas. I found research very different from my academic work.
While working on problem sets I typically do not know the answer but usually find I know potential
approaches to tackle the problem. In contrast, this research was challenging in that I both needed to
ask the right questions to best guide my ideas, as well as find appropriate techniques and methods that
answer those questions. This is not easy, and gaining this experience so early in my career
will surely aid me in the future. I would like to thank Northwestern and the Office of Undergraduate
Research for giving me this opportunity, as well as Dr.~Jason Siefken for his incredible mentorship and
support throughout this project, even when separated by great distances.
\end{document}
