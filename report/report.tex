\documentclass[12pt,letterpaper]{article}
\usepackage[margin=1in]{geometry}
\usepackage[style=numeric,citestyle=numeric-comp,backend=bibtex]{biblatex}
\usepackage{sectsty}
\usepackage{titlesec}
\usepackage{amsmath}

\sectionfont{\large}

\titlespacing\section{0pt}{12pt plus 4pt minus 2pt}{2pt plus 2pt minus 2pt}

% \bibliography{references}

\begin{document}
\section*{Introduction}
Fragility describes how a system is likely to respond to small perturbations to its parameters. However,
there currently is no quantification of fragility. Thus, its utility is limited. Throughout this project
I investigated the existence of a mathematical quantity that best captures the intuitions of fragility.
To do so, I considered stochastic models such as birth and death chains that have the potential to
describe a variety of scenarios, such as market fluctuations. Making heavy use of computer simulation to
guide my intuition, I explored the mathematical properties of a number of specific birth and death
chains through analytical methods.

\section*{Project Phases}
\noindent
\textbf{Mathematical Context:} At first, the conception of mathematical fragility seemed quite vague.
This was due to my inexperience with probability theory, the framework that supports the models
considered here. I needed a better understanding of the context, especially stochastic models. I worked
through exercises and referred to a number of textbooks, introducing myself to the key concepts of
recurrence and hitting-time. This took longer than originally expected, yet proved critically important
as I would have been left without the appropriate language or definitions necessary to consider birth
and death chains.

\noindent
\textbf{Initial Simulations:} I next turned my attention towards some specific probability
distributions. In particular, I needed to develop my intuitions about birth and death chains, and the
property of recurrence---which describes the chain's potential to return to a certain point. To do so, I
used Python with the \texttt{NumPy} and \texttt{Matplotlib} libraries to simulate birth and death chains
defined by various probability distributions, and analyze and plot recurrence times to $0$. I have
previous experience doing similar forms of data analysis, and found that this came naturally to me.
However, I found it challenging to set appropriate parameters for each simulation---too many data points
causes simulations to run unnecessarily long, but too few (or poor sampling technique) produces
inconclusive results. This phase was particularly guiding as I learned techniques that can be used to
quickly investigate hypotheses.

In particular, my first simulations led me to define the \emph{power-distributed chain} with parameter
$\beta$, whose behavior is governed by the probability $p(n)$ and $q(n)$, where $p(n)$ represents the
probability of the chain transitioning in the positive direction, and $q(n)$ the negative. This chain is
defined by
\[
    p(n) = \begin{cases}
        1 & \text{if } n = 0 \\
        \frac{1}{2+n^{-\beta}} & \text{if } n > 0
    \end{cases}, \quad \quad
    q(n) = 1-p(n).
\]
I hypothesized that the power-distributed chain is \emph{positive recurrent}---i.e. returns to $0$
infinitely often and in finite expected time---if $\beta < 1$, and is \emph{null recurrent}---i.e.
returns to $0$ infinitely often but with infinite expected return time---if $\beta \geq 1$. Simulations
allowed me to compare recurrence times over $\beta$ values and the results supported my hypothesis.

\noindent
\textbf{Analytic Proof:} In mathematics, however, numerical results from simulation do not constitute
sufficient evidence. As such, I turned to analytic methods to prove my conjecture about the
power-distributed chain. While I drew from a number of established results in my attempts at proof, I
remained challenged by the range of techniques that I utilized to establish convergence properties of
important sums. I often became frustrated if I felt like I was working in circles, and as such this
phase required my persistence as a mathematician. However, I was able to make some key observations that
allowed me to succeed in proving my conjecture. Overall, the time spent here feels too long for the
results established, but it challenged both my mathematical ability and my perserverance.

\noindent
\textbf{The Right Question:} Through my analytic investigation of the power-distributed birth and death
chain, I had to consider if recurrence properties offered the best information, or if other properties
about birth and death chains would be more enlightening. This led me to consider the more general
\emph{hitting time}---the expected number of steps for a chain in a particular state to reach a
different state. Hitting times are more general than recurrence times and better represent a system's
``efficiency'' in reaching a particular state, while recurrence times are a more direct representation
of whether a system frequently ``gets lost.'' These considerations refined my conceptions about how
these quantities relate to the physical behavior of a chain.

\noindent
\textbf{Exploration of Hitting Times:} As a reasonable starting point, I considered the behavior of
hitting times under variations in step-size. Thus, I was investigating the quantity $h_s(n)$, the
expected hitting time to $0$ from position $n$ with step size $s$. Unfortunately, I expected calculation
of hitting times to be far too difficult, and so had to investigate using alternative methods. I
returned to simulations, updating the necessary parameters to instead vary the step-size of the
power-distrbuted chain. This helped enlighten quantities of particular interest, such as $\partial\, h/
\partial\, s$, the change in hitting time with respect to step-size. I believe that these quantities
hold important information about the fragility of a system.



\section*{Further Research}

\section*{Final Thoughts}
Above all, this project helped me to better understand my strengths and limits as a mathematician. It
allowed me the invaluable opportunity to experience research early in my career. Math research, due to
the emphasis on abstraction, is difficult to predict, and this project required that I persist even when
I felt behind schedule or lacking in ideas. Further, this project tested the limits of my mathematical
creatitivity, an invaluable tool that I will surely use in future work and one that has been
significantly improved due to my work here. I would like to thank the Office of Undergraduate Research
for allowing me this opportunity as well as Jason Siefken for his incredible mentorship and support
throughout this project, even when separated by great distances.
\end{document}
